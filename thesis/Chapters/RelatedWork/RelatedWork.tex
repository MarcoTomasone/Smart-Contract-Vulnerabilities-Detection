\documentclass[../../Thesis.tex]{subfiles}
\usepackage[italian]{babel}

\begin{document}
\chapter{Lavori Correlati}
\label{ch:relatedwork}
Il tema della rilevazione delle vulnerabilità all'interno degli Smart Contracts è un tema che ha guadagnato notevole importanza nel tempo, anche a seguito della grande diffusione della tecnologia blockhain. Proprio per questo motivo, sono stati sviluppati e proposti diversi approcci per la rilevazione automatica di vulnerabilità all'interno degli Smart Contracts. In questo capitolo verranno presentati alcuni dei lavori più significativi che si sono occupati di questo tema.
Tra i principali approcci proposti figurano gli approcci basati su analisi statica basati su tecniche di esecuzione simbolica. L'analisi statica si basa sull'esame del codice sorgente o bytecode di uno smart contract senza effettuarne l'esecuzione effettiva. Questo approccio consente di identificare potenziali problematiche senza la necessità di testare il codice in un ambiente reale. L'esecuzione simbolica è una tecnica particolarmente potente in questo contesto in quanto consente di esplorare tutte le possibili esecuzioni del programma, consentendo di individuare vulnerabilità che potrebbero emergere solo in determinate condizioni.
Gli approcci basati su esecuzione simbolica cercano di risolvere queste vulnerabilità attraverso la generazione di un grafo di esecuzione simbolico, in cui le variabili sono rappresentate come simboli e le esecuzioni possibili del programma vengono esplorate simbolicamente. Ciò consente di identificare percorsi di esecuzione che potrebbero condurre a condizioni di errore o vulnerabilità.
Tuttavia, va notato che l'analisi statica, inclusa l'esecuzione simbolica, può essere complessa e non sempre completa. Alcune vulnerabilità potrebbero sfuggire a questa analisi o richiedere ulteriori tecniche di verifica. Pertanto, è consigliabile combinare l'analisi statica con altre metodologie, come l'analisi dinamica e i test formali, per garantire una copertura completa nella rilevazione di vulnerabilità negli smart contract. Tra i principali strumenti che utilizzano questo tipo di analisi ci sono Maian \cite{Maian, Maian2}, Oyente \cite{Oyente, Oyente2}, Mythril \cite{Mythril}, Manticore \cite{Manticore} e altri.
Un'altro tipo di approcci ad analisi statica sono i tools basati su regole. Questi strumenti usano un set di regole predefinite e pattern per identificare delle potenziali nvulnerabilità nel codice sorgente. Questi tool analizzano il codice sorgente e segnalano tutte le istanze dove il codice viola delle regole predefinite. Le regole sono tipicamente basate su delle vulnerabilità note e delle best practice di programmazione, come ad esempio evitare dei buffer overflow, usare algoritmi di cifratura sicuri e validare propriamente l'input degli utenti. La limitazione principale di questi strumenti è che i risultati che producono sono limitati al set di regole che è stato implementato, quindi non riescono a riconoscere delle nuove vulnerabilità o vulnerabilità non scoperte precedentemente. Inoltre, un'altra grande limitazione di questi strumenti è il fatto che possano produrre un alto numero di falsi positivi, cioè di situazioni in cui il codice viene segnalato come codice vulnerabile ma in realtà è codice perfettamente sano. Tra i principali strumenti che utilizzano questo tipo di analisi ci sono Slither \cite{Slither}, Securify \cite{Securify}, SmartCheck \cite{SmartCheck} e altri.






\end{document}