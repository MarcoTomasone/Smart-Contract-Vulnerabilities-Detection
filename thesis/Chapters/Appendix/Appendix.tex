\documentclass[../../Thesis.tex]{subfiles}
\usepackage[italian]{babel}

\begin{document}

\chapter*{Appendice}
\section*{Iperparametri del modello Alberi di Decisione}
Gli iperparametri scelti per il modello di decision tree sono stati: 
\begin{itemize}
    \item \textbf{Criterion}: gini
    \item \textbf{Max Depth}: 15
    \item \textbf{Min Samples Leaf}: 1
    \item \textbf{Min Samples Split}: 2
    \item \textbf{Splitter}: random
\end{itemize}

\section*{Iperparametri del modello Random Forest}

Gli iperparametri scelti per il modello di random forest sono stati:
\begin{itemize}
    \item \textbf{Max Depth}: 15
    \item \textbf{Min Samples Leaf}: 4
    \item \textbf{Min Samples Split}: 10
    \item \textbf{N Estimators}: 300
    \item \textbf{Bootstrap}: True
\end{itemize}
Per l'eliminazione del comportamento di overfitting, gli iperparametri scelti sono stati:
\begin{itemize}
    \item \textbf{Max Depth}: 10
    \item \textbf{Min Samples Leaf}: 2
    \item \textbf{Min Samples Split}: 10
    \item \textbf{N Estimators}: 100
    \item \textbf{Bootstrap}: True
\end{itemize}

\section*{Iperparametri del modello Support Vector Machine}
Gli iperparametri scelti per il modello di support vector machine sono stati:
\begin{itemize}
    \item \textbf{C}: 1
    \item \textbf{Kernel}: rbf
    \item \textbf{Gamma}: scale
\end{itemize}

\section*{Iperparametri del modello Gaussian NaiveBayes}
L'iperparametro scelto per il modello di Gaussian NaiveBayes è stato:
\begin{itemize}
    \item \textbf{Var Smoothing}: 1e-09
\end{itemize}

\section*{Iperparametri del modello Logistic Regression}
Gli iperparametri scelti per il modello di logistic regression sono stati:
\begin{itemize}
    \item \textbf{C}: 1
    \item \textbf{Max Iter}: 100
    \item \textbf{Solver}: liblinear
\end{itemize}
\end{document}