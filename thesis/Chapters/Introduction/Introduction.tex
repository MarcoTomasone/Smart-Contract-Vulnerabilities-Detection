\documentclass[../../Thesis.tex]{subfiles}
\usepackage[italian]{babel}

\begin{document}
\chapter{Introduzione}
Nel mondo delle tecnologie blockchain, gli smart contracts hanno guadagnato popolarità grazie alla loro abilità di automatizzare e far rispettare gli accordi tra due soggetti senza la necessità di un intermediario. 
Negli ultimi anni una delle tecnologie che sono spopolate e sono arrivate sulla bocca di tutti sono sicuramente la Blockchain e gli SmartContracts. Questi ultimi sono dei contratti digitali che permettono di eseguire delle operazioni in modo automatico e trasparente. Questi contratti sono scritti in un linguaggio di programmazione e vengono eseguiti su una macchina virtuale. Una delle principali caratteristiche peculiari degli SmartContracts è sicuramente la loro immutabilità, difatti una volta essere stati deployati sulla blockchain questi non possono più essere modificati.  Uno dei problemi principali degli SmartContracts è la sicurezza. Infatti, essendo dei contratti che vengono eseguiti in modo automatico, è possibile che ci siano delle vulnerabilità che possono essere sfruttate da malintenzionati. In questo lavoro di tesi verrà presentato un metodo per rilevare le vulnerabilità presenti negli SmartContracts. 
\end{document}